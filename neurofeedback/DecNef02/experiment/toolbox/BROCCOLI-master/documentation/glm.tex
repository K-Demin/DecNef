\section{GLM}

BROCCOLI provides a separate function for the general linear model (GLM). A GLM can for a single subject be performed as

\begin{verbatim}
GLM volumes.nii -design design.txt -contrasts contrasts.txt -firstlevel
\end{verbatim}
where volumes.nii is a 4D file containing all volumes from an experiment, design.txt contains all the raw regressors and can for example be defined as

\begin{verbatim}
NumRegressors 2

1.0 0.0
1.0 0.0
1.0 0.0
1.0 0.0
1.0 0.0
0.0 1.0
0.0 1.0
0.0 1.0
0.0 1.0
0.0 1.0
\end{verbatim}
while the contrasts.txt file can be defined as

\begin{verbatim}
NumRegressors 2
NumContrasts 2

1.0 -1.0
-1.0 1.0
\end{verbatim}
\newpage
\noindent For a single subject, the GLM can also be performed as 

\begin{verbatim}
GLM data.nii -designfiles regressors.txt -contrasts contrasts.txt -firstlevel
\end{verbatim}
where regressors.txt can look like

\begin{verbatim}
NumRegressors 3

task1.txt
task2.txt
task3.txt
\end{verbatim}
Each task file needs to include the number of events, the start time of each event, the length of each event and the value for the regressor of each event (this format is very similar to the one used by the FSL software). A task file can for example look like

\begin{verbatim}
NumEvents 8

47.532543       2.517515        1
68.789386       2.950905        1
93.547212       2.934332        1
165.670213      3.584591        1
190.078151      1.817203        1
257.349564      2.584177        1
280.457018      1.367038        1
308.66593       2.084071        1
\end{verbatim}
For all first level analyses, BROCCOLI will automatically add detrending regressors and remove the mean of all the activity regressors. \newline 

\subsection{Several runs}

If several runs are available for one subject, they can be analyzed together as
\begin{verbatim}
GLM -runs 3 volumes1.nii volumes2.nii volumes3.nii -design design1.txt 
design2.txt design3.txt -contrasts contrasts.txt -firstlevel [options]
\end{verbatim}
or
\begin{verbatim}
GLM -runs 3 volumes1.nii volumes2.nii volumes3.nii -designfiles regressors1.txt 
regressors2.txt regressors3.txt -contrasts contrasts.txt -firstlevel [options]
\end{verbatim}
If, for some reason, a subject did not do a specific task for one or several runs, this can be handled by using a dummy regressor, e.g.
\begin{verbatim}
NumRegressors 3

task1.txt
task2.txt
dummy.txt
\end{verbatim}
where dummy.txt only contains
\begin{verbatim}
NumEvents 0
\end{verbatim}

\subsection{Checking the design matrix}

To check if the design matrix looks correct, the options -savedesignmatrix and -saveoriginaldesignmatrix can be used. A text file will be generated with all the regressors (-savedesignmatrix), or only the task regressors (-saveoriginaldesignmatrix). If the AFNI software package is installed, these can for example be viewed as

\begin{verbatim}
1dplot volumes_total_designmatrix.txt
\end{verbatim}
or
\begin{verbatim}
1dplot volumes_original_designmatrix.txt
\end{verbatim}

\subsection{Group analysis}

\noindent For a group analysis, a GLM can be performed as
\begin{verbatim}
GLM volumes.nii -design design.txt -contrasts contrasts.txt -secondlevel
\end{verbatim}
where volumes.nii is a 4D file containing beta values from each subject.

\section{OpenCL options}

The following OpenCL options are available

\begin{itemize}

\item -platform
\newline \newline The OpenCL platform to use (default 0).

\item -device
\newline \newline The OpenCL device to use (default 0).

\end{itemize}

\section{GLM options}

The following GLM options are available

\begin{itemize}

\item -design
\newline \newline The design matrix to use in the GLM. 

\item -designfiles
\newline \newline The regressors to use in the GLM, each line in the text file points to a task file. 

\item -contrasts                 
\newline \newline The contrast vector(s) to apply to the estimated beta values. 

\item -firstlevel
\newline \newline Do a GLM for a single subject.
 
\item -secondlevel
\newline \newline Do a GLM for several subjects.
 
\item -teststatistics
\newline \newline Test statistics to use, 0 = GLM t-test, 1 = GLM F-test  (default 0).

\item -betasonly
\newline \newline Only save the beta values (no t- or F-test), contrast file not needed (default no)

\item -contrastsonly
\newline \newline Only save the contrast values (default no)

\item -betasandcontrastsonly
\newline \newline Only save the beta values and the contrast values (default no)

\item -mask                      
\newline \newline A mask that defines which voxels to analyze (default none). 
\end{itemize}


\section{Single subject options}

The following single subject options are available

\begin{itemize}

\item -runs
\newline \newline Number of fMRI runs that will be concatenated and analyzed together, necessary to provide one design file per run. Separate detrending regressors will be added for each run. The number of task regressors will not change compared to a single run, meaning that the contrast vectors are the same as for a single run.

\item -rawregressors
\newline \newline Use raw regressors (FSL format, one line per volume/TR). These regressors will not be convolved with any hemodynamic response function. Intended for -designfiles, not for -design.

\item -detrendingregressors
\newline \newline Set the number of detrending regressors (per run), 1 - 4. 1 means a single regressor for the intercept, 2 means intercept + linear trends, 3 means intercept + linear trends + quadratic trends, 4 means intercept + linear trends + quadratic trends + cubic trends (the default is 4 detrending regressors).

\item -temporalderivatives
\newline \newline Use temporal derivatives for the activity regressors (default no). The contrast vectors will automatically be extended with zeros for these additional regressors. Only possible for -designfiles, not for -design.

\item -regressmotion
\newline \newline Provide a file with motion regressors to use in the design matrix (default no). The file can be generated by FirstLevelAnalysis or MotionCorrection. The contrast vectors will automatically be extended with zeros for these additional regressors.

\item -regressglobalmean
\newline \newline Include global mean in design matrix (default no). The contrast vectors will automatically be extended with zeros for this additional regressor.


\end{itemize}


\section{Outputs}

By default, the GLM function saves each beta value as 

\begin{verbatim}
volumes_beta_regressor001.nii ... volumes_beta_regressor00R.nii
\end{verbatim}
for R regressors. For each contrast, BROCCOLI also saves the contrast of parameter estimate (COPE) and the corresponding t-scores, e.g.

\begin{verbatim}
volumes_cope_contrast001.nii  - volumes_cope_contrast00C.nii

volumes_tscores_contrast001.nii - volumes_tscores_contrast00C.nii
\end{verbatim}

\section{Output options}

The following output options are available

\begin{itemize}


\item -saveresiduals
\newline \newline Save the residuals (default no). 

\item -saveresidualvariance
\newline \newline Save the residual variance (default no).

\item -savearparameters
\newline \newline Save the estimated AR coefficients (first level only, default no).

\item -saveoriginaldesignmatrix
\newline \newline Save the original design matrix used (default no).

\item -savedesignmatrix
\newline \newline Save the total design matrix used (default no).

\item -output 
\newline \newline Set output filename (default volumes\_).

\end{itemize}

\section{Additional options}

The following additional options are available

\begin{itemize}

\item -quiet 
\newline \newline Don't print anything to the terminal (default false). 

\item -verbose
\newline \newline Print extra stuff (default false). 

\end{itemize}


