\section{ICA}

BROCCOLI provides a function for independent component analysis (ICA). In its simplest form, it can be used as

\begin{verbatim}
ICA volumes.nii
\end{verbatim}
where volumes.nii is a 4D file with several volumes.

\section{OpenCL options}

The following OpenCL options are available

\begin{itemize}

\item -platform
\newline \newline The OpenCL platform to use (default 0).

\item -device
\newline \newline The OpenCL device to use (default 0).

\end{itemize}

\newpage 

\section{ICA options}

The following ICA options are available

\begin{itemize}

\item -var
\newline \newline Proportion of variance to save before ICA (default 80 \%).  

\item -mask
\newline \newline Provide a spatial mask (default false), otherwise a mask will automatically be created.

\item -zscore
\newline \newline Z-score each time series before ICA (default false). 

\item -cpu
\newline \newline Use the CPU only (default false).

\item -double
\newline \newline Use double precision for all calculations, instead of single precision floats (default false). 

\end{itemize}

\section{Outputs}

By default, the function saves the independent components as volumes\_ica.nii.
\section{Output options}

The following output options are available

\begin{itemize}

\item -output 
\newline \newline Set output filename (default volumes\_ica.nii).

\end{itemize}

\section{Additional options}

The following additional options are available

\begin{itemize}

\item -quiet 
\newline \newline Don't print anything to the terminal (default false). 

\item -verbose
\newline \newline Print extra stuff (default false). 

\end{itemize}


